%%%%%%%%%%%%%%%%%%%%%%%%%%%%%%%%%%%%%%%%%%%%%%%%%%%%%%%%%%%%%%%%%%%%%%%%%%%%%%%%%%%%%%%%%%%%%%%%%%%%%%%%%%%%%%%%%%%%%%%%%%%%%%%%%%%%%%%%%%%%%%%%%%%%%%%%%%%%%%%%%%%
% Written By Michael Brodskiy
% Class: Circuits & Signals: Biomedical Applications
% Professor: N. Sun
%%%%%%%%%%%%%%%%%%%%%%%%%%%%%%%%%%%%%%%%%%%%%%%%%%%%%%%%%%%%%%%%%%%%%%%%%%%%%%%%%%%%%%%%%%%%%%%%%%%%%%%%%%%%%%%%%%%%%%%%%%%%%%%%%%%%%%%%%%%%%%%%%%%%%%%%%%%%%%%%%%%

\documentclass[12pt]{article} 
\usepackage{alphalph}
\usepackage[utf8]{inputenc}
\usepackage[russian,english]{babel}
\usepackage{titling}
\usepackage{amsmath}
\usepackage{graphicx}
\usepackage{enumitem}
\usepackage{amssymb}
\usepackage[super]{nth}
\usepackage{everysel}
\usepackage{ragged2e}
\usepackage{geometry}
\usepackage{multicol}
\usepackage{fancyhdr}
\usepackage{cancel}
\usepackage{siunitx}
\usepackage{physics}
\usepackage{tikz}
\usepackage{mathdots}
\usepackage{yhmath}
\usepackage{cancel}
\usepackage{color}
\usepackage{array}
\usepackage{multirow}
\usepackage{gensymb}
\usepackage{tabularx}
\usepackage{extarrows}
\usepackage{booktabs}
\usetikzlibrary{fadings}
\usetikzlibrary{patterns}
\usetikzlibrary{shadows.blur}
\usetikzlibrary{shapes}

\geometry{top=1.0in,bottom=1.0in,left=1.0in,right=1.0in}
\newcommand{\subtitle}[1]{%
  \posttitle{%
    \par\end{center}
    \begin{center}\large#1\end{center}
    \vskip0.5em}%

}
\usepackage{hyperref}
\hypersetup{
colorlinks=true,
linkcolor=blue,
filecolor=magenta,      
urlcolor=blue,
citecolor=blue,
}


\title{Steady-State Sinusoidal Analysis}
\date{\today}
\author{Michael Brodskiy\\ \small Professor: N. Sun}

\begin{document}

\maketitle

\begin{itemize}

  \item The voltage for a sinusoid may be found using:

    $$\boxed{v=V_m\cos(\omega t+\phi)}$$

  \item Where $V_m$ is the average voltage, $\omega$ is the angular frequency, and $\phi$ is the phase shift angle

  \item The two frequencies are:

    $$f=\frac{1}{T},\,\,\,\,\,\,\,\,\,\,\text{Frequency}$$
    $$\omega=\frac{2\pi}{T},\,\,\,\,\,\,\,\,\,\,\text{Angular Frequency}$$

  \item The mean value of a periodic signal is defined as:

    $$V_m=\frac{1}{T}\int_0^T v\,dt$$

    \begin{center}
      \textsc{OR}
    \end{center}

    \vspace{-15pt}

    $$\boxed{V_m=\int_0^TV_m\cos(\omega t+\phi)\,dt=0}$$

  \item The root mean square (rms) may be calculated as follows:

    $$V_{rms}=\sqrt{\frac{1}{T}\int_0^Tv^2\,dt}$$

    \begin{center}
      \textsc{OR}
    \end{center}

    \vspace{-15pt}

    $$\boxed{V_{rms}=\sqrt{\frac{1}{T}\int_0^T V_m^2\cos^2(\omega t+\phi)\,dt}=\frac{V_m}{\sqrt{2}}}$$

  \item Solving for the differential equation of sinusoidal response, current becomes:

    $$i=\dfrac{-V_m}{\sqrt{R^2+\omega^2L^2}}\cos(\phi-\theta)e^{-\dfrac{Rt}{L}}+\dfrac{V_m}{\sqrt{R^2+\omega^2L_^2}}\cos(\omega t+\phi-\theta)$$

  \item The left term is the transient response, and the right term is the steady-state response

  \item The transient will go to 0, so the steady-state term is most important

  \item Frequency of the output signal is the same as the frequency of the input signal

  \item Amplitude and phase changes (new phase is $\phi-\theta$)

  \item The Phasor Transform

    \begin{itemize}

      \item The phasor transform utilizes Euler's identity to rewrite the response

      \item The frequency domain becomes

        $$V=P\{V_m\cos(\omega t +\phi)i\}=V_me^{j\phi}e^{j\omega t}$$

      \item Where $V$ is the frequency domain

      \item $V=100\angle-26^{\circ}$ is the phasor representation of $100\cos(\omega t-25)$

      \item For an inverse phasor transform, multiply by $e^{j\omega t}$. The exponential becomes the term inside of the $\cos$. Convert.

    \end{itemize}

\end{itemize}

\end{document}

