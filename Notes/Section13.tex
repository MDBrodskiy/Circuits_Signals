%%%%%%%%%%%%%%%%%%%%%%%%%%%%%%%%%%%%%%%%%%%%%%%%%%%%%%%%%%%%%%%%%%%%%%%%%%%%%%%%%%%%%%%%%%%%%%%%%%%%%%%%%%%%%%%%%%%%%%%%%%%%%%%%%%%%%%%%%%%%%%%%%%%%%%%%%%%%%%%%%%%
% Written By Michael Brodskiy
% Class: Circuits & Signals: Biomedical Applications
% Professor: N. Sun
%%%%%%%%%%%%%%%%%%%%%%%%%%%%%%%%%%%%%%%%%%%%%%%%%%%%%%%%%%%%%%%%%%%%%%%%%%%%%%%%%%%%%%%%%%%%%%%%%%%%%%%%%%%%%%%%%%%%%%%%%%%%%%%%%%%%%%%%%%%%%%%%%%%%%%%%%%%%%%%%%%%

\include{Includes.tex}

\title{Fourier Transform}
\date{\today}
\author{Michael Brodskiy\\ \small Professor: N. Sun}

\begin{document}

\maketitle

\begin{itemize}

  \item The series can be represented as a sum of sinusoidal signals:

    $$f(t)=\sum_{-\infty}^{\infty} C_ne^{jn\omega_0t},\quad\quad\quad C_n=\frac{1}{T}\int_{-\frac{T}{2}}^{\frac{T}{2}} f(t)e^{-jn\omega_0 t}\,dt$$

  \item As $T\to\infty$, $\dfrac{1}{T}\to\dfrac{d\omega}{2\pi}$

  \item The Fourier transform is written as:

    $$F(\omega)=\mathcal{F}\{f(t)\}=\int_{-\infty}^{\infty}f(t)e^{-j\omega t}\,dt$$

\end{itemize}

\end{document}

