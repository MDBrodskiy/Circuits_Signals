%%%%%%%%%%%%%%%%%%%%%%%%%%%%%%%%%%%%%%%%%%%%%%%%%%%%%%%%%%%%%%%%%%%%%%%%%%%%%%%%%%%%%%%%%%%%%%%%%%%%%%%%%%%%%%%%%%%%%%%%%%%%%%%%%%%%%%%%%%%%%%%%%%%%%%%%%%%%%%%%%%%
% Written By Michael Brodskiy
% Class: Circuits & Signals: Biomedical Applications
% Professor: N. Sun
%%%%%%%%%%%%%%%%%%%%%%%%%%%%%%%%%%%%%%%%%%%%%%%%%%%%%%%%%%%%%%%%%%%%%%%%%%%%%%%%%%%%%%%%%%%%%%%%%%%%%%%%%%%%%%%%%%%%%%%%%%%%%%%%%%%%%%%%%%%%%%%%%%%%%%%%%%%%%%%%%%%

\documentclass[12pt]{article} 
\usepackage{alphalph}
\usepackage[utf8]{inputenc}
\usepackage[russian,english]{babel}
\usepackage{titling}
\usepackage{amsmath}
\usepackage{graphicx}
\usepackage{enumitem}
\usepackage{amssymb}
\usepackage[super]{nth}
\usepackage{everysel}
\usepackage{ragged2e}
\usepackage{geometry}
\usepackage{multicol}
\usepackage{fancyhdr}
\usepackage{cancel}
\usepackage{siunitx}
\usepackage{physics}
\usepackage{tikz}
\usepackage{mathdots}
\usepackage{yhmath}
\usepackage{cancel}
\usepackage{color}
\usepackage{array}
\usepackage{multirow}
\usepackage{gensymb}
\usepackage{tabularx}
\usepackage{extarrows}
\usepackage{booktabs}
\usetikzlibrary{fadings}
\usetikzlibrary{patterns}
\usetikzlibrary{shadows.blur}
\usetikzlibrary{shapes}

\geometry{top=1.0in,bottom=1.0in,left=1.0in,right=1.0in}
\newcommand{\subtitle}[1]{%
  \posttitle{%
    \par\end{center}
    \begin{center}\large#1\end{center}
    \vskip0.5em}%

}
\usepackage{hyperref}
\hypersetup{
colorlinks=true,
linkcolor=blue,
filecolor=magenta,      
urlcolor=blue,
citecolor=blue,
}


\title{Sinusoidal Instantaneous Power}
\date{\today}
\author{Michael Brodskiy\\ \small Professor: N. Sun}

\begin{document}

\maketitle

\begin{itemize}

  \item Instantaneous power is given by

    $$p=iv$$

  \item Where $v=V_m\cos(\omega t+\theta_v)$ and $i=I_m\cos(\omega t + \theta_i)$

  \item Changing the reference time for the sinusoidal, we can create:

    $$\left\{\begin{array}{c} v= V_m\cos(\omega t + \theta_v - \theta_i)\\ i=I_m\cos(\omega t)\end{array}$$

  \item This makes the instantaneous power:

    $$p=V_mI_m\cos(\omega t + \theta_v - \theta_i)\cos(\omega t)$$

    $$p=\frac{V_mI_m}{2}\cos(\theta_v-\theta_i)+\frac{V_mI_m}{2}\cos(\theta_v-\theta_i)\cos(2\omega t)-\frac{V_mI_m}{2}\sin(\theta_v-\theta_i)\sin(2\omega t)$$

  \item Or, to simplify:

    $$p=P+P\cos(2\omega t)-Q\sin(2\omega t)$$

  \item The average (real) power becomes $P=\frac{V_mI_m}{2}\cos(\theta_v-\theta_i)$

  \item The reactive power becomes $Q=\frac{V_mI_m}{w}\sin(2\omega t)$

  \item Units

    \begin{itemize}

      \item Instantaneous power unit is $\si{\volt\ampere}$ (Volt-Ampere)

      \item The average (real) power unit is watts

      \item The reactive power units is $\si{\volt\ampere}$R (Volt-Ampere Reactive)

    \end{itemize}

  \item Using root-mean square

    $$\left\{\begin{array}{c} P=V_{rms}I_{rms}\cos(\theta_v-\theta_i)\\ Q=V_{rms}I_{rms}\sin(\theta_v-\theta_i)\end{array}$$

    $$\frac{Q}{P}=\tan(\theta_v-\theta_i)$$

  \item Next, we use $|S|=\sqrt{P^2+Q^2}$ and $S=P+jQ$ to define a phasor transform

    $$S=V_{rms}I_{rms}\angle{(\theta_v-\theta_i)}$$

  \item In Euler notation:

    $$\frac{V_mI_m}{2}e^{j(\theta_v-\theta_i)}$$

\end{itemize}

\end{document}

