%%%%%%%%%%%%%%%%%%%%%%%%%%%%%%%%%%%%%%%%%%%%%%%%%%%%%%%%%%%%%%%%%%%%%%%%%%%%%%%%%%%%%%%%%%%%%%%%%%%%%%%%%%%%%%%%%%%%%%%%%%%%%%%%%%%%%%%%%%%%%%%%%%%%%%%%%%%%%%%%%%%
% Written By Michael Brodskiy
% Class: Circuits & Signals: Biomedical Applications
% Professor: N. Sun
%%%%%%%%%%%%%%%%%%%%%%%%%%%%%%%%%%%%%%%%%%%%%%%%%%%%%%%%%%%%%%%%%%%%%%%%%%%%%%%%%%%%%%%%%%%%%%%%%%%%%%%%%%%%%%%%%%%%%%%%%%%%%%%%%%%%%%%%%%%%%%%%%%%%%%%%%%%%%%%%%%%

\include{Includes.tex}

\title{Inductance and Capacitance}
\date{\today}
\author{Michael Brodskiy\\ \small Professor: N. Sun}

\begin{document}

\maketitle

\begin{itemize}

  \item What is an inductor?

    \begin{itemize}

      \item An electrical component that opposes change in electric current

        \begin{itemize}

          \item Unlike a resistor, which opposes the flow of current

        \end{itemize}

      \item Made by putting a coil of wire around a magnetic or non-magnetic core

      \item Source of inductance is change in magnetic field

      \item Current causes magnetic field and change in current causes change in magnetic field, which induces voltage in conductors (inductance)

      \item Mathematical Relation

        $$\boxed{v=L\frac{di}{dt}}$$

      \item $L$ is the inductance and its SI unit is Henry ($\si{\henry}$)

      \item Notice the direction of current and voltage drop

      \item Mathematical Relation\footnote{If voltage is given}

        $$\boxed{i(t)=\frac{1}{L}\int_{t_0}^t vdt+i(t_0)}$$

      \item Energy is given by $E=\frac{1}{2}Li^2$

    \end{itemize}

  \item What is a capacitor?

    \begin{itemize}

      \item Separation of charge produces voltage which causes electric field

      \item The amount of current produced by time varying electric field depends on the physical properties of dielectric materials

      \item This is called capacitance

    \end{itemize}

  \item Mathematical Relation for Capacitors

        $$\boxed{i=C\frac{dv}{dt}}$$

    \begin{itemize}

      \item Current is induced due to change in voltage with time

      \item $C$ is the conductance and its SI unit is Farad($\si{\farad}$)

      \item Notice the direction of current and voltage drop

      \item If current flows in opposite direction then there will be a minus sign in the equation

      \item The energy stored is $E=\frac{1}{2}cv^2$

    \end{itemize}

  \item Inductors in Series and Parallel

    \begin{itemize}

      \item For a serially connected inductor, the equivalent inductance is:

        $$\boxed{L_{eq}=L_1+L_2+\cdots+L_n}$$

      \item For inductors in parallel, the equivalent inductance is:

        $$\boxed{\dfrac{1}{L_{eq}}=\dfrac{1}{L_1}+\dfrac{1}{L_2}+\cdots+\dfrac{1}{L_n}}$$

    \end{itemize}

  \item Capacitors in Series and Parallel

    \begin{itemize}

      \item For a serially connected capacitor, the equivalent capacitance is:

        $$\boxed{\dfrac{1}{C_{eq}}=\dfrac{1}{C_1}+\dfrac{1}{C_2}+\cdots+\dfrac{1}{C_n}}$$

      \item For capacitors in parallel, the equivalent capacitance is:

        $$\boxed{C_{eq}=C_1+C_2+\cdots+C_n}$$

    \end{itemize}

  \item Summary Table

    \renewcommand{\arraystretch}{2.5}
  \begin{center}
    \begin{tabular}[H]{| l c c c |}
      \hline
      Property & $R$ & $L$ & $C$\\
      \hline
      $i-v$ relation & $i=\dfrac{v}{R}$ & $i=\dfrac{1}{L}\displaystyle \int_{t_0}^t v\,dt + i(t_0)$ & $i=C\dfrac{dv}{dt}$\\
      $v-i$ relation & $v=iR$ & $v=L\dfrac{di}{dt}$ & $v=\dfrac{1}{C}\displaystyle \int_{t_0}^t i\,dt+v(t_0)$\\
      $p$ (power transfer in) & $p=i^2R$ & $p=Li\dfrac{di}{dt}$ & $p=Cv\dfrac{dv}{dt}$\\
      $w$ (stored energy) & 0 & $w=\frac{1}{2}Li^2$ & $w=\frac{1}{2}Cv^2$\\
      Series Combination & $R_{eq}=R_1+R_2$ & $L_{eq}=L_1+L_2$ & $\dfrac{1}{C_{eq}}=\dfrac{1}{C_1}+\dfrac{1}{C_2}$\\
      Parallel Combination & $\dfrac{1}{R_{eq}}=\dfrac{1}{R_1}+\dfrac{1}{R_2}$ & $\dfrac{1}{L_{eq}}=\dfrac{1}{L_1}+\dfrac{1}{L_2}$ & $C_{eq}=C_1+C_2$\\
      DC Behavior & No change & Short circuit & Open circuit \\
      Instantaneous $v$ change? & Yes & Yes & No\\
      Instantaneous $i$ change? & Yes & No & Yes\\
      \hline
    \end{tabular}
  \end{center}

\end{itemize}

\end{document}

