%%%%%%%%%%%%%%%%%%%%%%%%%%%%%%%%%%%%%%%%%%%%%%%%%%%%%%%%%%%%%%%%%%%%%%%%%%%%%%%%%%%%%%%%%%%%%%%%%%%%%%%%%%%%%%%%%%%%%%%%%%%%%%%%%%%%%%%%%%%%%%%%%%%%%%%%%%%%%%%%%%%
% Written By Michael Brodskiy
% Class: Circuits & Signals: Biomedical Applications
% Professor: N. Sun
%%%%%%%%%%%%%%%%%%%%%%%%%%%%%%%%%%%%%%%%%%%%%%%%%%%%%%%%%%%%%%%%%%%%%%%%%%%%%%%%%%%%%%%%%%%%%%%%%%%%%%%%%%%%%%%%%%%%%%%%%%%%%%%%%%%%%%%%%%%%%%%%%%%%%%%%%%%%%%%%%%%

\include{Includes.tex}

\title{Node Voltage Method}
\date{\today}
\author{Michael Brodskiy\\ \small Professor: N. Sun}

\begin{document}

\maketitle

\begin{itemize}

  \item Node — A point where two or more circuit elements join

  \item Essential Node — A node where three or more circuit elements join

  \item Path — A trace of adjoining basic elements with no elements included more than once

  \item Branch — A path that connects two nodes

  \item Essential Branch — A path which connects two essential nodes without passing through an essential node

  \item Loop — A path whose last node is the same as the starting node

  \item Mesh — A loop that does not enclose any other loops

  \item Planar Circuit — A circuit that can be drawn on a plane with no crossing branches

  \item The Node Voltage Method — 4 Step Process
    
    \begin{enumerate}

      \item Find essential nodes (\#=$n$)

      \item Designate a ground node

      \item Write $n-1$ KCL equations, setting current flowing out of each essential node = 0 (not including the ground node)

      \item Solve $n-1$ equations

    \end{enumerate}

\end{itemize}

\end{document}

