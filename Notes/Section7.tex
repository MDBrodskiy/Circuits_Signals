%%%%%%%%%%%%%%%%%%%%%%%%%%%%%%%%%%%%%%%%%%%%%%%%%%%%%%%%%%%%%%%%%%%%%%%%%%%%%%%%%%%%%%%%%%%%%%%%%%%%%%%%%%%%%%%%%%%%%%%%%%%%%%%%%%%%%%%%%%%%%%%%%%%%%%%%%%%%%%%%%%%
% Written By Michael Brodskiy
% Class: Circuits & Signals: Biomedical Applications
% Professor: N. Sun
%%%%%%%%%%%%%%%%%%%%%%%%%%%%%%%%%%%%%%%%%%%%%%%%%%%%%%%%%%%%%%%%%%%%%%%%%%%%%%%%%%%%%%%%%%%%%%%%%%%%%%%%%%%%%%%%%%%%%%%%%%%%%%%%%%%%%%%%%%%%%%%%%%%%%%%%%%%%%%%%%%%

\include{Includes.tex}

\title{Signals, Systems, and Complex Numbers}
\date{\today}
\author{Michael Brodskiy\\ \small Professor: N. Sun}

\begin{document}

\maketitle

\begin{itemize}

  \item What is a System?

    \begin{itemize}

      \item From electrical engineering perspective, a system is device or a group device or a set-up that takes an input signal $x$ and manipulates it to generate a signal $y$

      \item Add another layer of abstraction over circuit (or other) models

      \item Can allow us to think about, analyze, and design circuits while paying attention only to what is important in a particular setting and ignoring lots of other details

    \end{itemize}

  \item Complex Numbers

    \begin{itemize}

      \item The imaginary numbers consist of all numbers $bi$, where $b$ is a real number and $i$ is the imaginary unit, with the property that $i^2=-1$

      \item The first four powers of $i$ establish an important pattern and should be memorized:

        $$\boxed{i^1=i\,\,\,\,\,\,\,\,\,\,\,\,i^2=-1\,\,\,\,\,\,\,\,\,\,\,\,i^3=-i\,\,\,\,\,\,\,\,\,\,\,\,i^4=1}$$

      \item The complex numbers consist of all sums $a + bi$, where $a$ and $b$ are real numbers and $i$ is the imaginary unit.  The real part is $a$, and the imaginary part is $bi$

      \item Conjugates

        \begin{itemize}

          \item The conjugate of $a+bi$ is $a-bi$

          \item The conjugate of $a-bi$ is $a+bi$

        \end{itemize}

    \end{itemize}

  \item Systems

    \begin{itemize}

      \item Physical System

        \begin{itemize}

          \item Defining a system involves drawing a boundary around some part of the world (or conceptually, inside a computational device) so that quantities external to those boundaries may influence what happens inside the boundaries (the ``inputs'' to the system)

        \end{itemize}

      \item Mathematical System

    \end{itemize}

  \item LTI Systems

    \begin{itemize}

      \item Linear System: Linear systems are systems whose outputs for a linear combination of inputs are the same as a linear combination of individual responses to those inputs

      \item Time Invariant System: Time-invariant systems are systems where the output does not depend on when an input was applied. These properties make LTI systems easy to represent and understand graphically

    \end{itemize}

  \item Fundamental Theorem of Algebra

    \begin{itemize}

      \item A polynomial of form $ax^n+bx^{n-1}+\cdots+c$ has $n$ roots

    \end{itemize}

  \item Complex Plane

    \begin{itemize}

      \item Any complex number can be placed on a 2D coordinate imaginary plane

      \item What would normal be the $y$ axis becomes the imaginary axis, and the $x$ axis acts as the real axis

    \end{itemize}

  \item Properties of Complex Numbers

    \begin{itemize}

      \item Instead of $i$, $j$ is used for complex numbers in electrical engineering because $i$ is used for current

      \item Euler's Formula:

        $$\boxed{e^{j\theta}=\cos(\theta)+j\sin(\theta)}$$

      \item The amplitude can be found using $c=\sqrt{a^2+b^2}$, where

        $$\boxed{\left\{\begin{array}{c c} a &= c\cos(\theta)\\b &= c\sin(\theta)\end{array}}$$ 

      \item Important properties:

        $$\boxed{a+jb=ce^{j\theta}}$$

        $$\boxed{\tan(\theta)=\frac{b}{a}}$$

      \item Rectangular Form:

          $$\boxed{n=a+jb}$$

      \item Polar Form:

          $$\boxed{n=ce^{j\theta}}$$

    \end{itemize}

\end{itemize}

\end{document}

