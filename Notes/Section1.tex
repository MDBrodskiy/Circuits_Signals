%%%%%%%%%%%%%%%%%%%%%%%%%%%%%%%%%%%%%%%%%%%%%%%%%%%%%%%%%%%%%%%%%%%%%%%%%%%%%%%%%%%%%%%%%%%%%%%%%%%%%%%%%%%%%%%%%%%%%%%%%%%%%%%%%%%%%%%%%%%%%%%%%%%%%%%%%%%%%%%%%%%
% Written By Michael Brodskiy
% Class: Circuits & Signals: Biomedical Applications
% Professor: N. Sun
%%%%%%%%%%%%%%%%%%%%%%%%%%%%%%%%%%%%%%%%%%%%%%%%%%%%%%%%%%%%%%%%%%%%%%%%%%%%%%%%%%%%%%%%%%%%%%%%%%%%%%%%%%%%%%%%%%%%%%%%%%%%%%%%%%%%%%%%%%%%%%%%%%%%%%%%%%%%%%%%%%%

\documentclass[12pt]{article} 
\usepackage{alphalph}
\usepackage[utf8]{inputenc}
\usepackage[russian,english]{babel}
\usepackage{titling}
\usepackage{amsmath}
\usepackage{graphicx}
\usepackage{enumitem}
\usepackage{amssymb}
\usepackage[super]{nth}
\usepackage{everysel}
\usepackage{ragged2e}
\usepackage{geometry}
\usepackage{multicol}
\usepackage{fancyhdr}
\usepackage{cancel}
\usepackage{siunitx}
\usepackage{physics}
\usepackage{tikz}
\usepackage{mathdots}
\usepackage{yhmath}
\usepackage{cancel}
\usepackage{color}
\usepackage{array}
\usepackage{multirow}
\usepackage{gensymb}
\usepackage{tabularx}
\usepackage{extarrows}
\usepackage{booktabs}
\usetikzlibrary{fadings}
\usetikzlibrary{patterns}
\usetikzlibrary{shadows.blur}
\usetikzlibrary{shapes}

\geometry{top=1.0in,bottom=1.0in,left=1.0in,right=1.0in}
\newcommand{\subtitle}[1]{%
  \posttitle{%
    \par\end{center}
    \begin{center}\large#1\end{center}
    \vskip0.5em}%

}
\usepackage{hyperref}
\hypersetup{
colorlinks=true,
linkcolor=blue,
filecolor=magenta,      
urlcolor=blue,
citecolor=blue,
}


\title{Introduction to Circuits \& Signals}
\date{\today}
\author{Michael Brodskiy\\ \small Professor: N. Sun}

\begin{document}

\maketitle

\begin{itemize}

  \item Electrical engineering deals with systems that produce, transmit, and measure electrical signals

    \begin{itemize}

      \item Electrical signals: Mostly voltages and currents

      \item Electrical components: Resistors, capacitors, inductors, etc.

        \begin{itemize}

          \item Resistor — Opposes the flow of current

          \item Capacitor — Stores the energy in electrical fields by storing charge to generate voltage

          \item Inductor — Stores energy in magnetic fields when current flows through it

        \end{itemize}

    \end{itemize}

  \item Communication systems deal with the generation, transmission, and distribution of information (Cable-TV, cellphone networks, old dial-up networks, radio telescopes, radar systems)

  \item What is an electric circuit?

    \begin{itemize}

      \item A complete or partial path over which current may flow

      \item Electric circuits consist of elements (voltage sources, current sources, resistors, capacitors, inductors, etc.)

      \item Electrons move in the conductors (wires, elements) in a circuit, giving current flow

      \item Can have moving electrons (-) or moving holes (+); protons and neutrons are immobile

    \end{itemize}

  \item Measurements

    \begin{itemize}

      \item Voltmeter — Measure voltage without drawing current

      \item Ammeter — Measures current without dropping voltage

    \end{itemize}

  \item The unit of charge is Coulomb (C)

    \begin{itemize}

      \item Charge can either be positive or negative

      \item The fundamental (smallest) quantity of charge is that of a single electron or proton. Its magnitude usually is denoted by the letter $e$ ($1.6\times10^{-19}$ C)

      \item According to the law of conservation of charge, the (net) charge in a closed region can neither be created nor destroyed

      \item Two like charges repel one another, whereas two charges of opposite polarity attract

    \end{itemize}

  \item Current flows in the direction opposite of electron flow

  \item $I=JA$, where $A$ is the cross-sectional area of a conductor, $J$ is the current density, and $I$ is the current

    \begin{itemize}

      \item $J=neu$, where $n$ is the density of charges, $e$ is the smallest magnitude of charge, and $u$ is the velocity of the electrons

    \end{itemize}

  \item Voltage and Current

    \begin{itemize}

      \item Voltage (V) is the electric potential difference of a point (in a circuit) relative to some other point (in the circuit)

      \item In a resistor: V, I, and R are related by Ohm's Law, where Resistance (R) is the electrical resistance to current flow

        $$V=IR$$

    \end{itemize}

  \item Resistance

    \begin{itemize}

      \item To calculate the resistance, we can use the formula $R=\dfrac{l}{\sigma A}=\rho\dfrac{l}{A}$, in ohms $\Omega$

    \end{itemize}

\end{itemize}

\end{document}

