\documentclass[
	letterpaper, % Paper size, specify a4paper (A4) or letterpaper (US letter)
	10pt, % Default font size, specify 10pt, 11pt or 12pt
]{CSUniSchoolLabReport}

%----------------------------------------------------------------------------------------
%	REPORT INFORMATION
%----------------------------------------------------------------------------------------

\title{Electrocardiogram\\ Circuits \& Signals \\ EECE2150} % Report title

\author{Michael \textsc{Brodskiy}}

\date{April 20, 2023} % Date of the report

%----------------------------------------------------------------------------------------


\begin{document}

\maketitle % Insert the title, author and date using the information specified above

\begin{center}
	\begin{tabular}{l r}
		Date Performed: & April 6, 2023 \\ % Date the experiment was performed
        Partner: & Juan \textsc{Zapata}, Oluwalaanu \textsc{Adeboye} \\ % Partner names
		Instructor: & Professor \textsc{Sun} % Instructor/supervisor
	\end{tabular}
\end{center}

\setcounter{section}{-1}

\section{Introduction}

The purpose of this laboratory experimentation was to unify all of the course concepts. Through integration of rudimentary circuit components like resistors, capacitors and inductors, in tandem with more complex components like operational amplifiers and band-pass filters, the goal was to construct a circuit that, when connected to three electrodes on the body, would generate a filtered electrocardiogram signal. Additionally, an analog-to-digital converter was then used to generate a heart rate plot from a sample.

\section{Discussion and Analysis}

\section{Conclusion}

Overall, this laboratory experiment allowed for an adequate unification of course concepts. By reiterating the basics and strengthening more advanced and, thus, difficult concepts, the lab creates a stronger understanding of all critical course concepts.

\end{document}
