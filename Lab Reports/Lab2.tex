\documentclass[
	letterpaper, % Paper size, specify a4paper (A4) or letterpaper (US letter)
	10pt, % Default font size, specify 10pt, 11pt or 12pt
]{CSUniSchoolLabReport}

%----------------------------------------------------------------------------------------
%	REPORT INFORMATION
%----------------------------------------------------------------------------------------

\title{Circuit Laws \\ Circuits \& Signals \\ EECE2150} % Report title

\author{Michael \textsc{Brodskiy}}

\date{January 23, 2023} % Date of the report

%----------------------------------------------------------------------------------------


\begin{document}

\maketitle % Insert the title, author and date using the information specified above

\begin{center}
	\begin{tabular}{l r}
		Date Performed: & January 12, 2023 \\ % Date the experiment was performed
        Partner: & Juan \textsc{Zapata} \\ % Partner names
		Instructor: & Professor \textsc{Sun} % Instructor/supervisor
	\end{tabular}
\end{center}

\setcounter{section}{-1}

\section{Introduction}

The performed experiment was done to better comprehend the calculations involved with simple DC circuits. By applying Kirchoff's Laws and Ohm's Law to real life concepts, a deep understanding of these laws is facilitated.

\section{A Very Simple DC Circuit}

\subsection{}

\begin{equation}
  \begin{split}
    \textit{Using Ohm's Law: } & 5[\si{\volt}]=I(1[\si{\kilo\ohm}])\\
    & I=.005[\si{\ampere}]=5[\si{\milli\ampere}]
  \end{split}
  \label{eq:1}
\end{equation}

\subsection{}

The obtained resistor is nearly $1[\si{\kilo\ohm}]$ (actual: $0.984[\si{\kilo\ohm}]$).

\subsection{}

The voltage drop across the resistor is $5[\si{\volt}]$. The current through the circuit is nearly $5[\si{\milli\ampere}]$ (actual: $5.009[\si{\milli\ampere}]$).

\subsection{}

Assuming the resistance is kept constant, if the voltage were doubled, the current would double as well, making it, in this case, $10[\si{\milli\ampere}]$. (Or, more precisely: $10.018[\si{\milli\ampere}]$)

\section{Kirchoff's Current Law}

\subsection{}

There should be approximately $5[\si{\milli\ampere}]$ of current across each resistor, and $10[\si{\milli\ampere}]$ through the whole circuit:

\begin{equation}
  \begin{split}
    i_1&=i_2+i_3\\
    5-&1000i_2=0\\
    5-&1000i_3=0\\
    i_2&=i_3=5[\si{\milli\ampere}]
  \end{split}
  \label{eq:2}
\end{equation}

\subsection{}

$R_1=0.984[\si{\kilo\ohm}]\approx1[\si{\kilo\ohm}]$ and $R_2=1[\si{\kilo\ohm}]$ exactly.

\subsection{}

The measurements in the actual circuit are as follows:

\begin{center}
  \begin{tabular}[h]{|c|c|}
    \hline
    $I_{tot}$ & $9.607[\si{\milli\ampere}]$\\
    \hline
    $I_1$ & $4.997[\si{\milli\ampere}]$\\
    \hline
    $I_2$ & $4.96[\si{\milli\ampere}]$\\
    \hline
  \end{tabular}
\end{center}

This satisfies Kirchoff's Current Law, as the estimated value should be around $10[\si{\milli\ampere}]$, and the given values are fairly close.

\subsection{}

The resistance of the new resistor is measured as $1.9704[\si{\kilo\ohm}]$, and the new total current is $7.447[\si{\milli\ampere}]$. The measured current through the old resistor, which is $1[\si{\kilo\ohm}]$, is $4.99[\si{\milli\ampere}]$. The measured current through the new resistor is $2.526[\si{\milli\ampere}]$. Logically, this satisfies Kirchoff's Current Law, as the new $2[\si{\kilo\ohm}]$ resistor should be receiving approximately half the current of the other resistor.

\section{Kirchoff's Voltage Law}

\subsection{}

Because the resistors are of equal resistance values, in series, and with no other components, each one should receive half of the voltage source, which, in this case, is $2.5[\si{\volt}]$. Overall, the voltage across both resistors should equal the sum of their individual voltage drops, or $5[\si{\volt}]$

\subsection{}

The true resistance values are $.9843[\si{\kilo\ohm}]$ for the first resistor, and $.9899[\si{\kilo\ohm}]$ for the second resistor.

\subsection{}

The measured voltages are $2.4939[\si{\volt}]$ for the first resistor, $2.5084[\si{\volt}]$ for the second resistor, and $5.002[\si{\volt}]$ for the voltage source. According to Kirchoff's Voltage Law, the voltage across each resistor should be:

\begin{equation}
  \begin{split}
    5-I-I=0\\
    I=2.5[\si{\ampere}]\\
  \end{split}
  \label{eq:3}
\end{equation}

This is fairly close to the measured values, and, as such, agrees with Kirchoff's Voltage Law.

\subsection{}

Changing one of the resistors from 1 to 2$[\si{\kilo\ohm}]$ would mean that the total resistance is now $3[\si{\kilo\ohm}]$. The more powerful resistor will receive $\frac{2}{3}$ of the voltage, while the other receives $\frac{1}{3}$. The observed values seem to agree with this:

\begin{center}
  \begin{tabular}[h]{|c|c|}
    \hline
    $V_1$ & $1.6695[\si{\volt}]$\\
    \hline
    $V_2$ & $3.335[\si{\volt}]$\\
    \hline
    $V_s$ & $5.003[\si{\volt}]$
    \hline
  \end{tabular}
\end{center}

\subsection{} \textcolor{green}{\checkmark}

\subsection{} Each resistor receives a voltage drop of $0[\si{\volt}]$ when disconnected, but $5[\si{\volt}]$ when the two are connected with a voltmeter. This means that the voltmeter has a non-zero resistance, which completes the circuit and permits current to flow.

\section{Conclusion}

This experiment demonstrates two important concepts: first and foremost, it confirms the accuracy of Kirchoff's Laws. Additionally, this experiment conveys the idea that, though the laws and formulas are accurate, there are many factors affecting the actual operation of circuits in real life, making the values vary slightly.

\end{document}
