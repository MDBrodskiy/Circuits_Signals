\documentclass[
	letterpaper, % Paper size, specify a4paper (A4) or letterpaper (US letter)
	10pt, % Default font size, specify 10pt, 11pt or 12pt
]{CSUniSchoolLabReport}

%----------------------------------------------------------------------------------------
%	REPORT INFORMATION
%----------------------------------------------------------------------------------------

\title{Introduction to RC Circuits \\ Circuits \& Signals \\ EECE2150} % Report title

\author{Michael \textsc{Brodskiy}}

\date{March 12, 2023} % Date of the report

%----------------------------------------------------------------------------------------


\begin{document}

\maketitle % Insert the title, author and date using the information specified above

\begin{center}
	\begin{tabular}{l r}
		Date Performed: & March 2, 2023 \\ % Date the experiment was performed
        Partner: & Juan \textsc{Zapata} \\ % Partner names
		Instructor: & Professor \textsc{Sun} % Instructor/supervisor
	\end{tabular}
\end{center}

\setcounter{section}{-1}

\section{Introduction}

The purpose of this laboratory experimentation is to familiarize oneself with circuits involving resistors, capacitors, and an operational amplifier. A circuit with a resistor, capacitor, and alternating current function generator are constructed to observe RC circuits behaving under alternating current.

\subsection{Which Frequency to Use} According to RC circuit rules, the best frequency to use would be $RC=20000(.1\cdot10^{-6})=2\cdot10^{-3}[\si{\second}]\rightarrow\frac{1}{10\cdot2\cdot10^{-3}}=50[\si{\hertz}]$

  \section{Part I}

  \subsection{Q1} From the waveform, $\tau$ would be:

  $$V(t)=V_oe^{-\dfrac{t}{\tau}}\rightarrow\tau=\frac{1.45}{\ln(1.94375)}=2.18[\si{\milli\second}]$$

  The value of $2.18[\si{\milli\second}]$ is close to the above value of $2[\si{\milli\second}]$

  \section{Part 2}

  \subsection{Q2} The magnitude of the signal generator is $2[\si{\volt}]$, with a cap of $1.3625[\si{\volt}]$

  \subsection{Q3} The following formula was used to convert between magnitude and phase:

  $$\frac{2\pi}{t_1}=\frac{x}{t_2}$$

  Where $t_1$ is the period for a full cycle, $t_2$ is the period for the newly-generated waveform, and $x$ is the phase. As an example, for $50[\si{\hertz}]$, the value was obtained as follows:

  $$\frac{2\pi}{12.5}=\frac{x}{1.7}\rightarrowx=.272\pi[\text{rad}]$$

  \begin{center}
    \begin{tabular}[h!]{l c r}
      Frequency ($\si{\hertz}$) & $\Delta$ Magnitude ($\si{\volt}$)& $\Delta$ Phase ($\pi$-rad)\\
      \hline
      .1 & 2.125 & .02\\
      1 & 2.125 & .02\\
      10 & 2.3 & .04\\
      100 & 1.15 & .318\\
      1000 & .16625 & .47\\
      10000 & .0305 & 5\\
    \end{tabular}
  \end{center}

  \subsection{Q4} As frequency is increased, the change in magnitude decreases, and the change in phase increases, and vice versa.

\section{Conclusion}

Overall, this laboratory experiment introduced us to the concept of RC circuits with alternating current. In such a manner, real-world examples were tested with theoretical formula applications.

\end{document}
